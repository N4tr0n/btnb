\documentclass[letterpaper]{article}
% Change "article" to "report" to get rid of page number on title page
\usepackage{fullpage}
\usepackage{amsmath,amsfonts,amsthm,amssymb}
\usepackage{setspace}
\usepackage{fancyhdr}
\usepackage{lastpage}
\usepackage{extramarks}
\usepackage{chngpage}
\usepackage{soul}
\usepackage[usenames,dvipsnames]{color}
\usepackage{graphicx,float,wrapfig}
\usepackage{ifthen}
\usepackage{listings}
\usepackage{courier}

\definecolor{MyDarkGreen}{rgb}{0.0,0.4,0.0}

% For faster processing, load Matlab syntax for listings
\lstloadlanguages{Matlab}%
\lstset{language=Matlab,
  frame=single,
  basicstyle=\small\ttfamily,
  keywordstyle=[1]\color{Blue}\bf,
  keywordstyle=[2]\color{Purple},
  keywordstyle=[3]\color{Blue}\underbar,
  identifierstyle=,
  commentstyle=\usefont{T1}{pcr}{m}{sl}\color{MyDarkGreen}\small,
  stringstyle=\color{Purple},
  showstringspaces=false,
  tabsize=5,
  % Put standard MATLAB functions not included in the default
  % language here
  morekeywords={xlim,ylim,var,alpha,factorial,poissrnd,normpdf,normcdf},
  % Put MATLAB function parameters here
  morekeywords=[2]{on, off, interp},
  % Put user defined functions here
  morekeywords=[3]{FindESS},
  morecomment=[l][\color{Blue}]{...},
  numbers=left,
  firstnumber=1,
  numberstyle=\tiny\color{Blue},
  stepnumber=5
}

% Includes a figure
% The first parameter is the label, which is also the name of the figure
% with or without the extension (e.g., .eps, .fig, .png, .gif, etc.)
% IF NO EXTENSION IS GIVEN, LaTeX will look for the most appropriate one.
% This means that if a DVI (or PS) is being produced, it will look for
% an eps. If a PDF is being produced, it will look for nearly anything
% else (gif, jpg, png, et cetera). Because of this, when I generate figures
% I typically generate an eps and a png to allow me the most flexibility
% when rendering my document.
% The second parameter is the width of the figure normalized to column width
% (e.g. 0.5 for half a column, 0.75 for 75% of the column)
% The third parameter is the caption.
\newcommand{\scalefig}[3]{
  \begin{figure}[ht!]
    % Requires \usepackage{graphicx}
    \centering
    \includegraphics[width=#2\columnwidth]{#1}
    %%% I think \captionwidth (see above) can go away as long as
    %%% \centering is above
    % \captionwidth{#2\columnwidth}%
    \caption{#3}
    \label{#1}
  \end{figure}}

% Includes a MATLAB script
% The first parameter is the label, which also is the name of the script
% without the .m.
% The second parameter is the optional caption.
\newcommand{\matlabscript}[2]
{\begin{itemize}\item[]\lstinputlisting[caption=#2,label=#1]{#1.m}\end{itemize}}
\makeatletter
% \newcommand{\rmnum}[1]{\romannumeral #1}
% \newcommand{\Rmnum}[1]{\expandafter\@slowromancap\romannumeral #1@}
\makeatother

% Personal commands
\newcommand{\iip}[2]{\ensuremath{\left[#1,#2\right]}}
\newcommand{\DeltaT}{\ensuremath{\tilde{\Delta}}}
\newcommand{\ThetaT}{\ensuremath{\tilde{\Theta}}}
\newcommand{\parens}[1]{\ensuremath{\left(#1\right)}}

%%%%%%%%%%%%%%%%%%%%%%%%%%%%%%%%%%%%%%%%%%%%%%%%%%%%%%%%%%%%% 
% Make title
\title{Bogolubov Transformations and Null Bases}
\date{\today}
\author{\textbf{Nathan Hitchings}}
%%%%%%%%%%%%%%%%%%%%%%%%%%%%%%%%%%%%%%%%%%%%%%%%%%%%%%%%%%%%% 

\begin{document}
\maketitle
\section{Introduction}
\section{Bogolubov Transformations}
Let $\mathfrak{V}$ be a complex vector space of possibly infinite
dimension. Suppose that on $\mathfrak{V}$ we have an inner product
$\iip{\cdot}{\cdot}:\mathfrak{V}\times\mathfrak{V}\to\mathbb{C}$ that
satisfies
\begin{equation}
  [f,g]=[g,f]^*
\end{equation}
and
\begin{equation}
  [f^*,g^*]=-[f,g]^*
\end{equation}
for any $f,g\in\mathfrak{V}$, where $*$ denotes complex
conjugation. Assume that 
$\{u_k,u_k^*\}\subset\mathfrak{V}$ and
$\{v_l,v_l^*\}\subset\mathfrak{V}$ are complete and
orthonormal. Further, assume that
$\{\Theta_m,\Delta_m\}\subset\mathfrak{V}$
and
$\{\tilde{\Theta}_n,\tilde{\Delta}_n\}\subset\mathfrak{V}$
are complete and null, i.e. that the following hold:

\begin{subequations}\label{Prop:grp}
\begin{gather}
  \iip{\Theta_m}{\Theta_{m^{\prime}}}=\iip{\Delta_m}{\Delta_{m^{\prime}}}=0\\
  \iip{\Theta_m}{\Delta_{m^\prime}}=-i\delta_{mm^\prime}\\
  \iip{\ThetaT_n}{\ThetaT_{n^{\prime}}}=\iip{\DeltaT_n}{\DeltaT_{n^{\prime}}}=0\\
  \iip{\ThetaT_n}{\DeltaT_{n^\prime}}=-i\delta_{nn^\prime}
\end{gather}
\end{subequations}

Since $\{\Theta_m,\Delta_m\}$ and
$\{\tilde{\Theta}_n,\tilde{\Delta}_n\}$ are
complete subsets of $\mathfrak{V}$ we can write any
$f\in\mathfrak{V}$ as a linear combination of the elements of these
sets. That is, for all $f\in\mathfrak{V}$ we have 
\begin{equation}
  f=\sum_m(a_m\Theta_m+b_m\Delta_m)=\sum_n(c_n\tilde{\Theta}_n+d_n\tilde{\Delta}_n)
\end{equation}
 where $a_m,b_m,c_n,d_n\in\mathbb{C}$. Using \eqref{Prop:grp} we find
 that
 \begin{align}
   a_m&=i[\Delta_m,f]&b_m&=i[\Theta_m,f]\\
   c_n&=i[\DeltaT_n,f]&d_n&=i[\ThetaT_n,f]
 \end{align}
 \subsection{Bogolubov Transformations between Null Bases}
It is well known that the Bogolubov transformations between
$\{u_k,u_k^*\}$ and $\{v_l,v_l^*\}$ are 
\begin{align}
  v&=\alpha u+\beta u^*\\
  u&=\alpha^{\dagger}v-\beta^Tv^*
\end{align}
where $\alpha$ and $\beta$ are matrices with entries
$\alpha_{lk}=[u_k,v_l]$ and
$\beta_{lk}=-[u_k^*,v_l]$ that are called Bogolubov
coefficients. We want to find the Bogolubov transformations between
$\{\Theta_m,\Delta_m\}$ and
$\{\tilde{\Theta}_n,\tilde{\Delta}_n\}$. We know
that they can be found because both sets are complete and so they can
be expanded in terms of each other. We have 
\begin{align}
  \ThetaT_n&=\sum_m(\gamma_{nm}\Theta_m+\rho_{nm}\Delta_m)\\
  \DeltaT_n&=\sum_m(\xi_{nm}\Theta_m+\eta_{nm}\Delta_m)
\end{align}
Using \eqref{Prop:grp} again we find that 
\begin{align}
  \gamma_{nm}&=i[\Delta_m,\ThetaT_n]&\rho_{nm}&=i[\Theta_m,\ThetaT_n]\\
  \xi_{nm}&=i[\Delta_m,\DeltaT_n]&\eta_{nm}&=i[\Theta_m,\DeltaT_n]
\end{align}
Finally, using matrix notation we have that 
\begin{align}
  \ThetaT&=\gamma\Theta+\rho\Delta\\
  \DeltaT&=\xi\Theta+\eta\Delta
\end{align}

Now, we need to find expressions for $\Theta$ and $\Delta$ in terms of $\ThetaT$ and
$\DeltaT$. To do this, we write
\begin{align}
\Theta_m&=\sum_n\left(e_{mn}\ThetaT_n+f_{mn}\DeltaT_n\right)\\
\Delta_m&=\sum_n\left(g_{mn}\ThetaT_n+h_{mn}\DeltaT_n\right)
\end{align}
Let's compute $e_{mn}$:
\begin{align}
\iip{\DeltaT_{n^\prime}}{\Theta_m}&=\iip{\DeltaT_{n^\prime}}{\sum_n\left(e_{mn}\ThetaT_n+f_{mn}\DeltaT_n\right)}\\
&=\iip{\DeltaT_{n^\prime}}{\sum_n e_{mn}\ThetaT_n}\\
&=e_{mn^\prime}\iip{\DeltaT_{n^\prime}}{\ThetaT_{n^\prime}}\\
&=-ie_{mn^\prime}
\end{align}
Thus,
\begin{equation}
e_{mn}=i\iip{\DeltaT_{n}}{\Theta_m}.
\end{equation}
If you look back to (12) you will notice that $e_{mn}=-\eta_{nm}^*$. If we were to
compute the rest of the coefficients we would see similar results. Hence, the
expressions we were looking for are, in matrix notation,
\begin{align}
\Theta&=-\eta^\dagger\ThetaT-\rho^\dagger\DeltaT\\
\Delta&=-\xi^\dagger\ThetaT-\gamma^\dagger\DeltaT
\end{align}
where $\dagger$ denotes conjugate transpose.

We can now, finally, compute the Bogolubov identities for two null, complete sets.
First, inserting (22) and (23) into (13) and (14) yields
\begin{align}
\ThetaT &= \gamma\parens{-\eta^\dag\ThetaT-\rho^\dag\DeltaT}+\rho\parens{-\xi^\dag-\gamma^\dag\DeltaT}\\
&=\parens{-\gamma\eta^\dag-\rho\xi^\dag}\ThetaT+\parens{-\gamma\rho^\dag-\rho\gamma^\dag}\DeltaT
\end{align}
and
\begin{align}
\DeltaT&=\xi\parens{-\eta^\dag\ThetaT-\rho^\dag\DeltaT}+\eta\parens{-\xi^\dag-\gamma^\dag\DeltaT}\\
&=\parens{-\xi\eta^\dag-\eta\xi^\dag}\ThetaT+\parens{-\xi\delta^\dag-\eta\gamma^\dag}\DeltaT
\end{align}
This clearly implies that
\begin{align}
\gamma\eta^\dag+\rho\xi^\dag&=-\mathrm{I}\\
\gamma\rho^\dag+\rho\gamma^\dag&=0
\end{align}
and
\begin{align}
\xi\eta^\dag+\eta\xi^\dagger&=0\\
\xi\rho^\dag+\eta\gamma^\dag&=-\mathrm{I}
\end{align}
Conversely, inserting (13) and (14) into (22) and (23) gives us
\begin{align}
\Theta&=\parens{-\eta^\dag\gamma-\rho^\dag\xi}\Theta+\parens{-\eta^\dag\rho-\rho^\dag\eta}\Delta\\
\Delta&=\parens{-\xi^\dag\gamma-\gamma^\dag\xi}\Theta+\parens{-\xi^\dag\rho-\gamma^\dag\eta}\Delta
\end{align}
Therefore,
\begin{align}
\eta^\dag\gamma+\rho^\dag\xi&=-\mathrm{I}\\
\eta^\dag\rho+\rho^\dag\eta&=0\\
\xi^\dag\gamma+\gamma^\dag\xi&=0\\
\xi^\dag\rho+\gamma^\dag\eta&=-\mathrm{I}
\end{align}
The equations (28)-(31) and (34)-(37) are the Bogolubov identities for two null,
complete sets.
\subsection{Bogolubov Transformations between Null and Orthonormal Bases}
The last thing we need to do is to find the Bogolubov transformations between a
complete, orthonormal set and a complete, null set. To this end, let $\{u_k,u_k^*\}$ be
complete and orthonormal and let $\{\Theta_n,\Delta_n\}$ be complete
and null. Let us also assume that $\Theta_n,\Delta_n\in\mathbb{R}$ for
all $n$. Then, $\Theta_n=\Theta_n^*$ and the same cam be said of
$\Delta_n$ for each $n$. Now, expanding each $\Theta_n$ and each
$\Delta_n$ in terms of $u_k$ and $u_k^*$ we get
\begin{align}
\Theta_n&=\sum\parens{\lambda_{nk}u_k+\lambda_{nk}^*u_k^*}\\
\Delta_n&=\sum\parens{\zeta_{nk}u_k+\zeta_{nk}^*u_k^*}
\end{align}
Using the inner product to isolate the coefficients $\lambda_{nk}$ and $\zeta_{nk}$ we
find that
\begin{align}
\lambda_{nk}&=\iip{u_k}{\Theta_n}\\
\zeta_{nk}&=\iip{u_k}{\Delta_n}
\end{align}
In matrix notation, (38) and (39) are
\begin{align}
\Theta&=\lambda u+\lambda^* u^*\\
\Delta&=\zeta u+\zeta^* u^*
\end{align}
Now, expanding $u_k$ in terms of $\Theta_n$ and $\Delta_n$ yields
\begin{equation}
u_k=\sum\parens{a_{kn}\Theta_n + b_{kn}\Delta_n}
\end{equation}
Again, using the inner product to isolate the coefficients we find
\begin{align}
a_{kn}&=i\iip{\Delta_n}{u_k}=i\zeta_{nk}^*\\
b_{kn}&=i\iip{\Theta_n}{u_k}=i\lambda_{nk}^*
\end{align}
Thus,
\begin{equation}
u=i\zeta^\dag\Theta+i\lambda^\dag\Delta
\end{equation}
Computing the Bogolubov identities in the same way as before we find that
\begin{align}
\lambda\zeta^\dag-\lambda^*\zeta^T&=-i\mathrm{I}\\
\lambda\lambda^\dag-\lambda^*\lambda^T&=0\\
\zeta\zeta^\dag-\zeta^*\zeta^T&=0\\
\zeta\lambda^\dag-\zeta^*\lambda^T&=-i\mathrm{I}\\
\zeta^\dag\lambda+\lambda^\dag\zeta&=-i\mathrm{I}\\
\zeta^\dag\lambda^*+\lambda^\dag\zeta^*&=0
\end{align}
\end{document}

%%%%%%%%%%%%%%%%%%%%%%%%%%%%%%%%%%%%%%%%%%%%%%%%%%%%%%%%%%%%% 
